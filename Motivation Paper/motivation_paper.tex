
\documentclass{beamer} 


\mode<presentation>
{
  \usetheme[hideothersubsections]{Berkeley}
  % or ...

  \setbeamercovered{transparent}
  % or whatever (possibly just delete it)
}

\usepackage{tikz}
\usepackage{graphicx}
\usepackage[english]{babel}
% or whatever

\usepackage[utf8]{inputenc}
% or whatever

\usepackage{times}
\usepackage[T1]{fontenc}
% Or whatever. Note that the encoding and the font should match. If T1
% does not look nice, try deleting the line with the fontenc.


\title[Open Policy Analysis ] % (optional, use only with long paper titles)
{Why We Need Open Policy Analysis}

\subtitle
{}

\author[] % (optional, use only with lots of authors)
{Fernando~Hoces de la Guardia\inst{1}\\
Sean~Grant\inst{2}\\
Edward Miguel\inst{1}}
% - Give the names in the same order as the appear in the paper.
% - Use the \inst{?} command only if the authors have different
%   affiliation.

\institute[Universities of Somewhere and Elsewhere] % (optional, but mostly needed)
{
  \inst{1}%
  UC Berkeley:\\
  Berkeley Initiative for Transparency in the Social Sciences\\
  \inst{2}%
  RAND\\
}
% - Use the \inst command only if there are several affiliations.
% - Keep it simple, no one is interested in your street address.

\date[] % (optional, should be abbreviation of conference name)
{Congressional Budget Office, March 2018\\
{\small Slides at} \url{http://www.github.com/fhoces/CBO2018}}
% - Either use conference name or its abbreviation.
% - Not really informative to the audience, more for people (including
%   yourself) who are reading the slides online

\subject{Research Transparency}
% This is only inserted into the PDF information catalog. Can be left
% out. 

\pgfdeclareimage[height=2cm]{university-logo}{../Images/BITSSlogo.png}
\logo{\pgfuseimage{university-logo}}

% If you have a file called "university-logo-filename.xxx", where xxx
% is a graphic format that can be processed by latex or pdflatex,
% resp., then you can add a logo as follows:

% \pgfdeclareimage[height=0.5cm]{university-logo}{university-logo-filename}
% \logo{\pgfuseimage{university-logo}}



% Delete this, if you do not want the table of contents to pop up at
% the beginning of each subsection:
%\AtBeginSubsection[]
%{
%  \begin{frame}<beamer>{Outline}
%    \tableofcontents[currentsection,currentsubsection]
%  \end{frame}
%}


% If you wish to uncover everything in a step-wise fashion, uncomment
% the following command: 

\beamerdefaultoverlayspecification{<.->}


\begin{document}

\begin{frame}
  \titlepage
\end{frame}




% Structuring a talk is a difficult task and the following structure
% may not be suitable. Here are some rules that apply for this
% solution: 

% - Exactly two or three sections (other than the summary).
% - At *most* three subsections per section.
% - Talk about 30s to 2min per frame. So there should be between about
%   15 and 30 frames, all told.

% - A conference audience is likely to know very little of what you
%   are going to talk about. So *simplify*!
% - In a 20min talk, getting the main ideas across is hard
%   enough. Leave out details, even if it means being less precise than
%   you think necessary.
% - If you omit details that are vital to the proof/implementation,
%   just say so once. Everybody will be happy with that.
%%%%%%%%%%%%%%%%%%%%%%%%%%%%%%%%%%%%%%%%%%%%%%%%%%%%%%%%%%%%%%%%%%%%%%%
%%%%%%%%%%%%%%%%%%%%%%%%%%%%%%%%%%%%%%%%%%%%%%%%%%%%%%%%%%%%%%%%%%%%%
\begin{frame}{Outline}
  \tableofcontents
  % You might wish to add the option [pausesections]
\end{frame}
 
 
\section[Evidence Based]{Policy Analysis And The Evidence-Based Policy Movement}

\begin{frame}{Policy Analysis And The Evidence-Based Policy Movement}

\end{frame} 

\section[Crisis in Research]{Reproducibility Crisis In Empirical Research}

\begin{frame}{Reproducibility Crisis In Empirical Research}


\end{frame} 

\section[Open Science]{The Open Science Movement}

\begin{frame}{The Open Science Movement}


\end{frame} 

\section[Crisis in PA]{Credibility Crisis Of Policy Analysis}

\begin{frame}{Credibility Crisis Of Policy Analysis}

\end{frame} 

\section[Open PA]{Open Science In Policy Analysis}

\begin{frame}{Open Science In Policy Analysis}
\begin{itemize}
\item Reproducibility
\item Analytic Transparency
\item Output Transparency 
\end{itemize}
\end{frame} 

\section{Challenges And Suggestions}

\begin{frame}{Challenges And Suggestions}

\end{frame} 

\section{Conclusions}

\begin{frame}{Conclusions}
\begin{itemize}
\item TOP for PA
\item Case Studies
\end{itemize}
\end{frame} 

\begin{frame}[noframenumbering]
\begin{center}
\vspace*{4em}
{\Large Thank you.\\}
\bigskip
Pre-print: [PP URL HERE]   \\
Contact:  \href{mailto:fhoces@berkeley.edu}{fhoces@berkeley.edu}

\end{center}
\end{frame}

\end{document}

